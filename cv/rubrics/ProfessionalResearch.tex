%%%%%%%%%%%%%%%%%%%%%%%%%%%%%%%%%%%%%%%%%%%%%%%%%%%%%%%%%%%%%%%%%% \iffalse %%%%
%                                                                              %
%  Copyright (c) 2014 - Michiel Helvensteijn   (www.mhelvens.net)              %
%                                                                              %
%  Please don't modify and distribute the LaTeX code of my personal CV.        %
%  The cv.tex file and the rubric files are not released under an open         %
%  source license. That's only the my-cv.cls and my-whenprinting.sty files.    %
%                                                                              %
%  I include these files merely as a usage example.                            %
%                                                                              %
%%%%%%%%%%%%%%%%%%%%%%%%%%%%%%%%%%%%%%%%%%%%%%%%%%%%%%%%%%%%%%%%%%%%%%% \fi %%%%

\begin{rubric}{Professional Research}
	\in*{Nov~~2018} FTE in DeepMotion: In order to apply the deep learning algorithm and models to practical products, I have two main tasks. One is model compression, include pruning, distillation, quantization, layer fusion, and NAS. I'm familiar with 8-bit quantization based on QNNPACK package of Caffe2, created by Facebook. The other is deployment, I mainly use TX2 and Xavier of NVIDIA’s frame, TDA3x of TI’s frame, and HUAWEI's Altas200DK. I'm familiar with CUDA and CUDNN programming, and Caffe source code, which are very important on deployment.
	\in*{Jul~~2018} Internship in MSRA: Auto ML and graph embedding. The class label and the neighborhood context
		prediction in a graph, with semi-supervised learning. Submitted a paper to IJCAI 2019, under reviewing.
	\fromTo{2017}{2018} Final Year Project of Undergradate:
		Developed a face recognition system based on deep Learning using Convolutional Neural Network(CNN) for
		commercial application, which has a good performance in both face comparison and identity verification, and can be utilized on many occasions.
	\in{Jun~~2017}
        Hui-Chun Chin and Tsung-Dao Lee Chinese Undergraduate Research Endowment(CURE) Program: Photonic Crystals. Optical Signal Dynamic Buffer.Guided by Prof. Chun Jiang.
\end{rubric}
